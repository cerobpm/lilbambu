\documentclass{article}
\usepackage[T1]{fontenc}
\usepackage{textcomp}

%%  Latex generated from POD in document odm_load.pm
%%  Using the perl module Pod::LaTeX
%%  Converted on Mon Aug 13 19:51:54 2018


\usepackage{makeidx}
\makeindex


\begin{document}

%% \tableofcontents

\section{NAME\label{NAME}\index{NAME}}


odm\_load package

\section{SYNOPSIS\label{SYNOPSIS}\index{SYNOPSIS}}
\begin{verbatim}
        use odm_load;
        my $dbh=odm_load::dbConnect;
        my %params=("VariableCode"=>"2","VariableName"=>"Gage height","VariableUnitsID"=>52);
        my @opts=("-U");
        my $status=odm_load::addVariable($dbh,\%params,\@opts);
\end{verbatim}
\section{DESCRIPTION\label{DESCRIPTION}\index{DESCRIPTION}}
\begin{verbatim}
        Este modulo sirve para insertar, editar y seleccionar registros de la base de datos ODM/PGSQL
\end{verbatim}
\subsection*{FUNCION dbConnect()\label{FUNCION_dbConnect_}\index{FUNCION dbConnect()}}
\begin{verbatim}
        $_[0] => SCALAR   configuration file location .ini 
        $_[1] => HASH (DBUSER=>username,DBPASSWORD=>pass) (opcional, override conf file)
\end{verbatim}
\subsubsection*{returns\label{returns}\index{returns}}
\begin{verbatim}
         database handle object (from DBI->connect)
\end{verbatim}
\subsection*{FUNCION addVariables()\label{FUNCION_addVariables_}\index{FUNCION addVariables()}}
\begin{description}

\item[{\$\_[0] =$>$ database handle object (from}] \textbf{DBI-$>$connect)}
\item[{\$\_[1] =$>$ HASHREF parametros: ( "VariableCode"=$>$}] \textbf{"10" , "VariableName"=$>$ "Discharge","VariableUnitsID"=$>$36)}
\item[{\$\_[2] =$>$ ARRAYREF options -U =$>$ on}] \textbf{conflict action do update}\end{description}
\subsubsection*{returns\label{returns}\index{returns}}
\begin{verbatim}
        {"status":"200 OK","VariableID":"$inserted_variable_id"} o {"status":"400 Bad Request"}
\end{verbatim}
\subsection*{funcion columnTypeCheck()\label{funcion_columnTypeCheck_}\index{funcion columnTypeCheck()}}
\begin{verbatim}
        $_[0] => Column Names ARRAY
        $_[1] => Column Types ARRAY
        $_[2] => input Columns HASH
        $_[3] => 1:allRequired, 2:checkValid, 0:none
\end{verbatim}
\subsubsection*{returns\label{returns}\index{returns}}
\begin{verbatim}
        HASH of valid Column names=>{types}
\end{verbatim}
\subsection*{funcion GetVariables()\label{funcion_GetVariables_}\index{funcion GetVariables()}}
\begin{verbatim}
        $_[0] => database connection handler
        $_[1] => parameters HASH [valid params=  VariableCode"=>"STRING","VariableName"=>"STRING","VariableUnitsID"=>"INTEGER", "SampleMedium"=>"STRING","ValueType"=>"STRING","IsRegular"=>"BOOLEAN","DataType"=>"STRING", "GeneralCategory"=>"STRING","TimeSupport"=>"INTEGER","TimeUnitsID"=>"INTEGER
        $_[2] => options ARRAY [valid opts -f=[wml,json,fwt,csv]  ]
\end{verbatim}
\subsubsection*{returns\label{returns}\index{returns}}
\begin{verbatim}
        VariablesResponse STRING in requested format (wml,json,fwt,csv) o {"status":"400 Bad Request"}
\end{verbatim}
\subsection*{funcion addSource()\label{funcion_addSource_}\index{funcion addSource()}}
\begin{verbatim}
        $_[0] => database connection handler
        $_[1] => parameters HASH [valid params=  "SourceID"=>"INTEGER","Organization"=>"STRING","SourceDescription"=>"STRING","SourceLink"=>"STRING","ContactName"=>"STRING","Phone"=>"STRING", "Email"=>"STRING","Address"=>"STRING","City"=>"STRING","State"=>"STRING","ZipCode"=>"STRING","Citation"=>"STRING","MetadataID"=>"INTEGER"
        $_[2] => options ARRAY [valid opts -U]  ]
\end{verbatim}
\subsubsection*{returns\label{returns}\index{returns}}
\begin{verbatim}
        {"status":"200 OK","SourceID":"$inserted_source_id"} o {"status":"400 Bad Request"}
\end{verbatim}
\subsection*{funcion GetSources()\label{funcion_GetSources_}\index{funcion GetSources()}}
\begin{verbatim}
        $_[0] => database connection handler
        $_[1] => parameters HASH [valid params=  "SourceID"=>"INTEGER","Organization"=>"STRING","SourceDescription"=>"STRING","SourceLink"=>"STRING","ContactName"=>"STRING","Phone"=>"STRING", "Email"=>"STRING","Address"=>"STRING","City"=>"STRING","State"=>"STRING","ZipCode"=>"STRING","Citation"=>"STRING","MetadataID"=>"INTEGER"
        $_[2] => options ARRAY [valid opts -f [wml,json,fwt,csv]  ]
\end{verbatim}
\subsubsection*{returns\label{returns}\index{returns}}
\begin{verbatim}
        ArrayOfSourceInfo STRING in requested format (wml,json,fwt,csv) o {"status":"400 Bad Request"}
\end{verbatim}
\subsection*{funcion makeRestRequest()\label{funcion_makeRestRequest_}\index{funcion makeRestRequest()}}
\begin{verbatim}
        $_[0] => database connection handler
        $_[1] => parameters HASH [valid params=  "Organization"=>"STRING"*,"RequestName"=>"STRING"*,"method"=>"STRING"       *:required
        $_[2] => options ARRAY [valid opts -f]  ]
\end{verbatim}
\subsubsection*{returns\label{returns}\index{returns}}
\begin{verbatim}
        HTTP Response content STRING in requested format (wml) o {"status":"400 Bad Request"}
\end{verbatim}
\subsection*{funcion GetSourceLink()\label{funcion_GetSourceLink_}\index{funcion GetSourceLink()}}
\begin{verbatim}
        $_[0] => database connection handler
        $_[1] => Organization STRING
        $_[2] => options ARRAY [valid opts -U]
\end{verbatim}
\subsubsection*{returns\label{returns}\index{returns}}
\begin{verbatim}
        STRING SourceLink
\end{verbatim}
\subsection*{funcion addSite()\label{funcion_addSite_}\index{funcion addSite()}}
\begin{verbatim}
        $_[0] => database connection handler
        $_[1] => parameters HASH [valid params=  "Organization"=>"STRING"*,"RequestName"=>"STRING"*,"method"=>"STRING"       *:required
        $_[2] => options ARRAY [valid opts -U]  ]
\end{verbatim}
\subsubsection*{returns\label{returns}\index{returns}}
\begin{verbatim}
        {"status":"200 OK","SiteID":"$inserted_site_id"} o {"status":"400 Bad Request"}
\end{verbatim}
\subsection*{funcion addSites()\label{funcion_addSites_}\index{funcion addSites()}}
\begin{verbatim}
        $_[0] => database connection handler
        $_[1] => sites ARRAY  
        $_[2] => options ARRAY [valid opts -U]  ]
\end{verbatim}
\subsubsection*{returns\label{returns}\index{returns}}
\begin{verbatim}
        {"status":"200 OK","SiteID":"$inserted_site_ids"} o {"status":"400 Bad Request"}
\end{verbatim}
\subsection*{funcion GetSites\label{funcion_GetSites}\index{funcion GetSites}}
\begin{verbatim}
        $_[0] => database connection handler
        $_[1] => parameters HASH [valid params=  "SiteID"=>"INTEGER","SiteCode"=>"STRING","SiteName"=>"STRING","north"=>"FLOAT","south"=>"FLOAT","east"=>"FLOAT","west"=>"FLOAT","SiteType"=>"STRING","State"=>"STRING","County"=>"STRING"]
        $_[2] => options ARRAY [valid opts -f]  ]
\end{verbatim}
\subsubsection*{returns\label{returns}\index{returns}}
\begin{verbatim}
        siteResponse STRING in requested format  or {"status":"400 Bad Request"}
\end{verbatim}
\subsection*{funcion GetSiteInfo()\label{funcion_GetSiteInfo_}\index{funcion GetSiteInfo()}}
\begin{verbatim}
        $_[0] => database connection handler
        $_[1] => parameters HASH [valid params=  "SiteCode"=>"STRING"]
        $_[2] => options ARRAY [valid opts -f]  ]
\end{verbatim}
\subsubsection*{returns\label{returns}\index{returns}}
\begin{verbatim}
        SitesResponse STRING in requested format  or {"status":"400 Bad Request"}
\end{verbatim}
\subsection*{funcion addValues\label{funcion_addValues}\index{funcion addValues}}
\begin{verbatim}
        $_[0] => database connection handler
        $_[1] => parameters HASH [valid params=  "Values"=>"ARRAY"*,"SiteID"=>"INTEGER"*,"VariableID"=>"INTEGER"*,"MethodID"=>"INTEGER","SourceID"=>"INTEGER"*,"QualityControl"=>"INTEGER","UTCOffset"=>"INTEGER"]    *:required
        $_[2] => options ARRAY [valid opts -U]  ]
\end{verbatim}
\subsubsection*{returns\label{returns}\index{returns}}
\begin{verbatim}
        {"status":"200 OK","ValuesID":[valueID_1,ValuesID2...]}  or {"status":"400 Bad Request"}
\end{verbatim}
\subsection*{funcion GetValues\label{funcion_GetValues}\index{funcion GetValues}}
\begin{verbatim}
        $_[0] => database connection handler
        $_[1] => parameters HASH [valid params=  "SiteCode"=>"STRING"*,"VariableCode"=>"STRING"*,"StartDate"=>"STRING"*,"EndDate"=>"STRING"*]    *:required
        $_[2] => options ARRAY [valid opts -f]  ]
\end{verbatim}
\subsubsection*{returns\label{returns}\index{returns}}
\begin{verbatim}
        timeSeriesResponse STRING in required format  or {"status":"400 Bad Request"}
\end{verbatim}
\subsection*{funcion parseWML()\label{funcion_parseWML_}\index{funcion parseWML()}}
\begin{verbatim}
        $_[0] => database connection handler
        $_[1] => parameters  input=WMLSTRING* o file=FILENAME* , source=STRING* 
        $_[2] => options ARRAY [valid opts -f]  ]
\end{verbatim}
\subsubsection*{returns\label{returns}\index{returns}}
\begin{verbatim}
        200 Ok or 400 Bad Request ; STRING in required format  si se indica -f wml,json,fwt,csv
=cut
\end{verbatim}


sub parseWML
\{
	my \$str;
	if(ref(\$\_[1]) ne "HASH") \{
		die "$\backslash$\$\_[1] debe ser un HASH$\backslash$n";
	\}
	if(defined \$\_[1]-$>$\{input\}) \{
		\$str=\$\_[1]-$>$\{input\};
	\} elsif(defined \$\_[1]-$>$\{file\}) \{
		open(my \$file,\$\_[1]-$>$\{file\}) or die "No se pudo abrir el archivo " . \$\_[1]-$>$\{file\} . " para lectura$\backslash$n";
		while($<$\$file$>$) 
		\{
			\$str .= \$\_;
		\}
	\} else \{
		die "Falta parametro input o file$\backslash$n";
	\}
	if(!defined \$\_[1]-$>$\{source\}) \{
		die "Falta parametro source$\backslash$n";
	\}
	\#
	\# LEE OPCIONES
	\#
	my \%opts;
	if(defined \$\_[2]) \{
		if(ref(\$\_[2]) ne "ARRAY") \{
			die "$\backslash$\$\_[2] debe ser ARRAY ref, pero es" . ref(\$\_[2]) . ".";
		\}
		foreach(@\$\_[2]) \{
			\$\_ =\texttt{\~{}} s/\^{}-+//g;
			if(\$\_ =\texttt{\~{}} /\^{}($\backslash$.+)=($\backslash$.+)\$/) \{
				\$opts\{\$1\}=\$2;
			\} else \{
				\$opts\{\$\_\}=1;
			\}
		\}
		\#\texttt{\~{}} \%opts= map \{ \$\_ =$>$ 1 \} @\{\$\_[2]\};
	\}
	\#\texttt{\~{}} print STDERR \$str . "$\backslash$n";
	my \$data;
	eval \{
		\$data = XML::LibXML-$>$load\_xml(string=$>$\$str);
	\} or do \{
		die "No se pudo parsear la respuesta pues no es XML valido$\backslash$n";
	\};
	my \$format = (defined \$opts\{f\}) ? \$opts\{f\} : "pg";
	if(\$data-$>$exists('//variables')) \{
		my \$allres = "";
		\#\texttt{\~{}} print STDERR "variables Exists$\backslash$n";
		my @variables = \$data-$>$find('//variables')-$>$[0]-$>$childNodes();
		\#\texttt{\~{}} if(@variables $>$ 0) \{
		my \$n=0;
		foreach my \$var (@variables) \{
			my \$literal = \$var-$>$to\_literal;
			print STDERR "varnum:\$n,literal:\$literal$\backslash$n";
			my \%params;
			my \%validColumns = ("variableCode"=$>$"STRING","variableName"=$>$"STRING","variableDescription"=$>$"STRING", "unitsName"=$>$"STRING", "unitsType"=$>$"STRING","sampleMedium"=$>$"STRING","valueType"=$>$"STRING","isRegular"=$>$"BOOLEAN","dataType"=$>$"STRING", "generalCategory"=$>$"STRING","timeSupport"=$>$"INTEGER","timeUnitsID"=$>$"INTEGER");
			foreach my \$key (keys \%validColumns) \{
				my \$xpath=(\$key =\texttt{\~{}} /units/) ? ( "./unit/" . \$key . "[1]" ) : "./" . \$key . "[1]";
				if(\$var-$>$exists(\$xpath)) \{
					my \$node = \$var-$>$findnodes(\$xpath)-$>$[0];
					my \$val = \$node-$>$textContent;
					\$params\{\$key\} = (\$key eq "variableCode") ? \$\_[1]-$>$\{source\} . ":" .  \$val : \$val;
					\#\texttt{\~{}} print STDERR "varnum:\$n;key:\$key, val:\$val$\backslash$n";
				\}
			\}
			if(\$format eq "pg") \{
				if(defined \$params\{variableCode\} and defined \$params\{variableName\} and defined \$params\{unitsName\} and defined  \$params\{unitsType\}) \{ 
					my \%upperparams = map \{ ucfirst(\$\_) =$>$ \$params\{\$\_\} \} keys \%params;
					my \$UnitsID = odm\_load::GetUnitsID(\$\_[0],$\backslash$\%upperparams);
					\#\texttt{\~{}} print STDERR "unitsID:\$UnitsID, variableName:" . \$upperparams\{VariableName\} . "$\backslash$n";
					\#\texttt{\~{}} next;
					\$upperparams\{VariableUnitsID\}=\$UnitsID;
					delete \$upperparams\{UnitsName\};
					delete \$upperparams\{UnitsType\};
					delete \$upperparams\{VariableDescription\};
					my @opts=("-U");
					my \$res = odm\_load::addVariable(\$\_[0],$\backslash$\%upperparams,$\backslash$@opts);
					\$allres .= \$res;
				\}
			\}
			\$n++;
		\}
		if(\$allres eq "") \{ 
			print STDERR "No se encontraron registros$\backslash$n";
			return 1;
		\} else \{
			return \$allres . "$\backslash$n";
		\} 
	\}
\}

\subsection*{funcion GetUnitsID()\label{funcion_GetUnitsID_}\index{funcion GetUnitsID()}}
\begin{verbatim}
        $_[0] => database connection handler
        $_[1] => parameters  UnitsName=STRING* UnitsType=STRING*
\end{verbatim}
\subsubsection*{returns\label{returns}\index{returns}}
\begin{verbatim}
        UnitsID INTEGER   or 349
\end{verbatim}
\subsection*{funcion addSeries()\label{funcion_addSeries_}\index{funcion addSeries()}}
\begin{verbatim}
        $_[0] => database connection handler
        $_[1] => parameters HASH [valid params=  "SiteCode"=>"STRING"*,"VariableCode"=>"STRING"*,"SourceCode"=>"STRING"*,"MethodCode"=>"STRING","MethodDescription"=>"STRING","MethodLink"=>"STRING","Organization"=>"STRING","SourceDescription"=>"STRING","citation"=>"STRING","qualityControlLevelCode"=>"STRING","qualityControlLevelDefinition"=>"STRING","valueCount"=>"INTEGER","beginDateTime"=>"STRING","endDateTime"=>"STRING","beginDateTimeUTC"=>"STRING","endDateTimeUTC"=>"STRING"       *:required
        $_[2] => options ARRAY [valid opts -U]  ]
\end{verbatim}
\subsubsection*{returns\label{returns}\index{returns}}
\begin{verbatim}
        {"status":"200 OK","SiteID":"$inserted_series_id"} o {"status":"400 Bad Request"}
\end{verbatim}
\subsection*{funcion insertDataValues\label{funcion_insertDataValues}\index{funcion insertDataValues}}
\begin{verbatim}
        $_[0] => database connection handler
        $_[1] => TimeSeries ARRAY of HASHES
        $_[2] => SiteCode varchar
        $_[3] => VariableCode varchar
        $_[4] => SourceCode varchar
        $_[5] => options ARRAY [valid opts -U]  ]
\end{verbatim}
\subsubsection*{returns\label{returns}\index{returns}}
\begin{verbatim}
        {"status":"200 OK","ValuesID":[valueID_1,ValuesID2...]}  or {"status":"400 Bad Request"}
\end{verbatim}
\printindex

\end{document}
